Notre projet de programmation Quizzly nous a permis d'appliquer de nombreuses connaissances acquises à la fac, telles que la gestion de bases de données, la programmation web avec les sockets et l'utilisation de schémas découverts en modélisation de niveau 2 pour la rédaction de ce rapport. Nous avons été ravis de pouvoir mettre en pratique ces compétences et de voir comment elles pouvaient être appliquées dans un projet concret.\\

Lors la première partie du projet, nous avons pris des décisions judicieuses qui se sont avérées scalables et qui ont permis de gagner du temps et des efforts pendant la deuxième partie du projet. Par exemple, nous avons choisi d'utiliser une base de données au lieu de fichiers textes, contrairement à ce que suggérait la présentation. Ce choix nous a permis de gérer les données plus facilement et d'effectuer des opérations plus complexes sur ces données. Nous pouvons prendre pour exemple ici les requêtes effectuées pour la partie statistiques du site.\\

Cependant, cette deuxième partie du projet nous a donné plusieurs défis.
Il fallait d'une part gérer la coordination des membres de l'équipe et faire attention au respect des délais, mais aussi de l'autre part, anticiper les contrôles continus et les devoirs demandés par la fac. 
Cette tâche n'était pas facile, car elle exigeait de planifier le travail à long terme tout en respectant les délais à court terme. 
Malgré cela, nous avons réussi à définir les priorités et les échéances pour chaque tâche, ce qui nous a permis de livrer un projet fonctionnel et de qualité dans les délais impartis.\\

En fin de compte, notre expérience sur Quizzly nous a appris beaucoup de choses sur la façon de travailler en équipe pour créer une application complexe. Nous avons découvert ce qui fonctionnait bien pour nous, et nous avons identifié des domaines dans lesquels nous pourrions nous améliorer pour devenir encore plus efficaces à l'avenir.